
\theoremstyle{definition}
\newtheorem{definicion}{Definici\'on}

\begin{frame}
\titlepage
\end{frame}

\begin{frame}
\frametitle{A cubrir hoy\ldots}
\tableofcontents
% You might wish to add the option [pausesections]
\end{frame}

\section{Selecci\'on de art\'iculos de configuraci\'on}
\begin{frame}
	\frametitle{Art\'iculos y conglomerados}
	
La identificaci\'on de art\'iculos es la selecci\'on, creaci\'on y especificaci\'on de: 
	\begin{itemize}
		\item Productos entregados al cliente
		\item Productos de trabajo internos determinados
		\item Productos adquiridos
		\item Herramientas, y otros activos del ambiente de trabajo del proyecto
		\item Otros art\'iculos usados en crear y describir estos productos de trabajo
	\end{itemize}
\end{frame}

\begin{frame}
	\frametitle{Seleccionando art\'iculos de configuraci\'on}
	La selecci\'on de art\'iculos bajo el control de la configuraci\'on 
	ocurre al principio de un proyecto, cuando se ha acordado cuales ser\'an
	los entregables. 

	Continua a lo largo del proyecto. 

	Es similar a la identificaci\'on de los objeto	s en durante un an\'alisis orientado a objetos. 

	No es posible usar algoritmos. 
\end{frame}

\begin{frame}
\frametitle{Algunos criterios para la selecci\'on de art\'iculos}
?`Que incluir en la configuraci\'on de software?
	\begin{itemize}
		\item Productos de trabajo que ser\'an usados por uno o m\'as grupos. 
		\item Productos de trabajo que cambiar\'an en el tiempo, a causa de errores o 
			cambios de requerimientos
		\item Productos de trabajo con dependencias, de tal manera que un cambio en uno 
			force un cambio en el otro
		\item Productos de trabajo que sean cr\'iticos para el proyecto. 
	\end{itemize}
\end{frame}
\begin{frame}
	\frametitle{Ejemplos de art\'iculos de configuraci\'on}
	\begin{itemize}
		\item Descripciones de procesos
		\item Requerimientos
		\item Dise\~no
		\item C\'odigo fuente
		\item Planes de pruebas y procedimientos
		\item Resultados de pruebas
		\item Descripcion de interfaces.
	\end{itemize}
\end{frame}

\begin{frame}
	\frametitle{Un par de ejemplos}
	Un desarrollo de una aplicaci\'on de WEB en Java.

	Un desarrollo de un device driver para Linux en C. 
\end{frame}
\begin{frame}
	En una aplicaci\'on de WEB 
	\begin{columns}
		\begin{column}{0.5\textwidth}
			\begin{itemize}
				\item Documento de an\'alisis de requerimientos
				\item Diagramas de Dise\~no de objetos
				\item Manual de usuario
				\item Archivos JAR de objetos no desarrollados por el equipo
				\item Archivos WAR
				\item Bases de datos
				\item C\'odigo fuente
			\end{itemize}
		\end{column}
		\begin{column}{0.5\textwidth}
			\begin{itemize}
				\item Archivos ANT
				\item Servidor de aplicaciones				\item JVM
				\item Certificados de seguridad
				\item Casos de uso
				\item Ayuda en linea
				\item Documentos de dise\~no de sistemas.
				\item \ldots
			\end{itemize}
		\end{column}
	\end{columns}
\end{frame}

\begin{frame}
	\frametitle{Despues de identificar ?`Que?}
Una vez completada la identificaci\'on de art\'iculos de configuraci\'on:
	\begin{itemize}
		\item Se le asigna un identificador \'unico. 
		\item Se especifican las caracter\'isticas de cada art\'iculo de configuraci\'on. 
		\item Especificar cuando se incluye el art\'iculo de configuraci\'on en el control de cambios
		\item Identificar al due\~no de cada art\'iculo de configuraci\'on
		\item Identificar relaciones entre los art\'iculos. 
	\end{itemize}
\end{frame}
\begin{frame}
	\frametitle{Establesciendo un sistema de administraci\'on de configuraci\'on}
	\begin{itemize}
		\item Establescer un mecanismo para manejar niveles m\'ultiples de control
		\item Establescer un sistema de control para asegurar accesos autorizados al CM
		\item Almacenar los art\'iculos de configuraci\'on en el sistema
	\end{itemize}
\end{frame}
\begin{frame}
	\frametitle{Baseline}
 	Una parte del equipo forma el Change Control Board. Ellos determinar\'an las versiones iniciales de 
	los art\'iculos de configuraci\'on necesarios para crear una linea base inicial. 

	La linea base inicial (baseline) es la colecci\'on de art\'iculos de configuraci\'on acordada por el CCB
	a partir de la cual el producto empezar\'a a evolucionar. 

	En un acuerdo general el baseline es liberado por el CCB
\end{frame}
\
\section{Administraci\'on de promociones}
\begin{frame}
	\frametitle{Promociones}
	\begin{itemize}
		\item A fin de cambiar un art\'iculo de configuraci\'on, el responsable primero usa una promoci\'on. 
		\item Una promoci\'on es la manera de hacer un cambio a un art\'iculo de configuraci\'on y hacerlo disponible de manera interna.
		\item Los cambios a un art\'iculo de configuraci\'on pueden estar sujetos a otros procesos (vgr. revisiones, pruebas unitarias, juntas de aprobaci\'on)
		\item Una vez que un art\'iculo de configuraci\'on es cambiado, su versi\'on cambia. Los cambios ser\'an permanentes. El art\'iculo de configuraci\'on tendr\'a una historia.  
	\end{itemize}
\end{frame}

\section{Administraci\'on de liberaciones}
\begin{frame}
	\frametitle{Liberaciones}
	\begin{itemize}
		\item Las liberaciones son creadas, de acuerdo a desiciones administrativas. 
		\item Se manejan de manera similar a una promoci\'on, pero requieren de una mayor coordinaci\'on. 
		\item Se sincronizan los diferentes aspectos del producto, de manera que las diferentes dependencias se satisfagan.
		\item Un equipo de control de calidad o validaci\'on asegura que la colecci\'on de art\'iculos del conglomerado sea consistente. 
	\end{itemize}
\end{frame}

\section{Administraci\'on de ramas}
\begin{frame}
	\frametitle{?'Por que hacer ramas?}
	Las ramificaciones son creadas para poder aislar \'areas de trabajo entre desarrolladores o equipos de trabajo. 
	\begin{enumerate}
		\item Los equipos de trabajo acuerdan durante cuanto tiempo se har\'a la ramificaci\'on. 
		\item Los equipos de trabajo acuerdan que interfaces cambiar y cuales conserver
		\item Se realiza el trabajo independiente. 
		\item Se une el trabajo entre ramas. 
	\end{enumerate}
	
\end{frame}
\begin{frame}
	\frametitle{Uniendo ramas de trabajo}
	\begin{itemize}
		\item Identificar areas comunes.
		\item Hacer uniones frecuentemente. 
		\item Comunicar conflictos posbibles. 
		\item Minimizar cambios en la rama principal. 
		\item Minimizar el n\'umero de ramas. 
	\end{itemize}
\end{frame}

\section{Administraci\'on de variantes}
\begin{frame}
	\frametitle{Un solo producto, muchas variaciones}
Una variante, es c\'odigo coexistente en diferentes variacion que el principal. 
	\begin{itemize}
		\item Diferentes \alert{sabores} de un mismo producto (simple, profesional, de lujo)
		\item Diferentes plataformas para un producto (android, iOS, Windows, Linux, MacOSx)
		\item Diferentes clientes, con una configuraci\'on especial. 
	\end{itemize}
\end{frame}

\begin{frame}
	\frametitle{Como manejar las variaciones}
	\begin{description}
		\item [Equipos de trabajo redundantes] Los requerimientos son entregados a un equipos de trabajos redundantes. 
			Estos trabajan de manera aislada para entregar los requerimientos.
		\item [Un proyecto \'unico] El c\'odigo com\'un es compartido como m\'odulos de conglomerado de configuraci\'on. 
			Equipos de trabajo se comprometen a respetar interfaces, y los cambios entre sistemas tienden a ser
			con bajo acoplamiento. El c\'odigo com\'un es frecuentemente llamado el n\'ucleo. 
			
	\end{description}
\end{frame}
\begin{frame}
	\frametitle{Problemas de la variaciones}
	\begin{description}
		\item [Un solo proveedor, m\'ultiples clientes] Los requerimientos entre los diferentes clientes pueden ir 
			difiriendo e incluso ser contradictorios. 
		\item [Tardanzas en el cumplimiento de requerimientos] La validaci\'on tiene que contemplar todos los 
			posibles escenarios. 
		\item [Inconsistencias entre plataformas] Las diferentes plataformas pueden causar limitaciones en 
			los posibles sistemas. 
	\end{description}
\end{frame}	
\section{Administraci\'on de cambios}
\begin{frame}
	\frametitle{Justificando los cambios}
	Es necesario documentar la raz\'on de los cambios. En teor\'ia, ning\'un art\'iculo de configuraci\'on debiera ser 
	cambiado sin una razon. 
	La administraci\'on de cambios es el control de estos cambios, documentando su raz\'on. 
        En general se sujeta a estos pasos: 
	\begin{enumerate}
		\item El cambio es requerido (se identifica quien lo solicita, la raz\'on (defecto, mejora)
		\item El cambio es analizado, por un consejo, o el desarrollo de proyecto.  
		\item El cambio es aprobado o rechazado.
		\item En caso de ser aprobado, se le da una prioridad se le asigna un due\~no, y es implementado. 
		\item El cambio es auditado. 
	\end{enumerate}
\end{frame}

\section{Sumario}
\begin{frame}
\end{frame}
