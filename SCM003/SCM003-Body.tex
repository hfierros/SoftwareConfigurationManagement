\theoremstyle{definition}
\newtheorem{definicion}{Definici\'on}

\begin{frame}
\titlepage
\end{frame}

\begin{frame}
\frametitle{A cubrir hoy\ldots}
\tableofcontents
% You might wish to add the option [pausesections]
\end{frame}

\section{IEEE-828}
\subsection{El standard IEEE-828}
\begin{frame}
	\frametitle{?`Cual es el contenido de un plan de administraci\'on de configurac\'on?}
	Seg\'un el standard IEEE-828: 
	\begin{enumerate}
		\item Introducci\'on
		\item Administraci\'on de SCM
		\item Actividades de SCM
		\item Programa de SCM
		\item Recursos de SCM
		\item Mantenimiento del plan. 
	\end{enumerate}
\end{frame}
\subsection{Introducci\'on}
\begin{frame}
	\frametitle{Secci\'on 1: Introducci\'on}
	Provee un sumario de las actividades de SCM, a f\'in de dar una 
	visi\'on general del plan. 

	Debe incluir cuatro temas:
	\begin{itemize}
		\item Prop\'osito del plan
		\item Alcance del plan. 
		\item Definici\'on de t\'erminos
		\item Referencias. 
	\end{itemize}
\end{frame}
\begin{frame}
	\frametitle{Ejemplo...}
	Introducci\'on

	El prop\'osito de este documento es comunicar el plan de administraci\'on de configuraci\'on 
	de software del proyecto "App para dominar al mundo" de los laboratorios ACME. 
	Este documento es una herramienta, que ser\'a tratada como un documento vivo, que se 
	evolucionar\'a durante las fases de desarrollo del proyecto. 
\ldots

\end{frame}

\subsection{Administraci\'on de SCM}
\begin{frame}
	\frametitle{Administraci\'on de SCM
	}
	Esta secci\'on contesta a la pregunta ?`Quien? 

	Consta de las siguientes partes:
	\begin{itemize}
		\item Organizaci\'on del proyecto
		\item Responsabilidades
		\item Referencias
	\end{itemize}
\end{frame}
\begin{frame}
	\frametitle{Administraci\'on del SCM: Organizaci\'on}
	El plan debe identificar:
	\begin{enumerate}
		\item Las unidades organizacionales que participan o son
			responsables de cualquier actividad de SCM.
		\item Los roles de estos miembros
		\item Las relaciones entre estos miembros
	\end{enumerate}

	Los miembros pueden ser proveedores, clientes, grupos, o individuos en
	la organizaci\'on
\end{frame}
\begin{frame}
	\frametitle{Administraci\'on del SCM: Responsabilidades}
	Describe una relaci\'on entre las actividades de la siguiente secci\'on y 
	y las unidades organizacionales de la secci\'on anterior. 

	Para los consejos y grupos de trabajo que se forman debe describir: 
	\begin{enumerate}
		\item Prop\'osito y objetivos. 
		\item Membres\'ia y afiliaci\'on
		\item Periodo de efectividad
		\item Alcance de autoridad
		\item Procedimientos operacionales
	\end{enumerate}
\end{frame}
\begin{frame}
	\frametitle{Administraci\'on del SCM: Referencias}
En esta secci\'on se describen las relaciones del plan con otras pol\'iticas, directivas y
procedimientos que afecten o limiten el plan. 
\end{frame}
\subsection{Actividades de SCM}
\begin{frame}
	\frametitle{Actividades de SCM}
Esta secci\'on documenta las actividades de SCM:
	\begin{enumerate}
		\item Identificaci\'on de la configurac\'on.
		\item Control de la configuraci\'on
		\item Estado de la configuraci\'on y reporteo
		\item Revisiones de la configuraci\'on. 
		\item Control de interfaces
		\item Control de proveedores. 
	\end{enumerate}
\end{frame}
\begin{frame}
	\frametitle{Identificaci\'on de la configuraci\'on}
	\begin{description}
		\item [Identificaci\'on de art\'iculos de la configurac\'on] 
			El plan registra los art\'iculos de configuraci\'on (AC) del 
			proyecto, sus definiciones, y c\'omo se mantendran por 
			el proyecto. Adem\'as define las lineas base en el ciclo
			de vida del proyecto. 
		\item [Nombrando los art\'iculos de configuraci\'on.] 
			El plan decribe los m\'etodos para nombrar los AC e 
			identificarlos. 
	\end{description}
\end{frame}
\begin{frame}
	\frametitle{Control de la configurac\'on}
	El plan describe como cambiar los AC en la l\'inea base.
	\begin{enumerate}
		\item Identificaci\'on y documentaci\'on de la necesidad del cambio.
		\item An\'alisis y evaluaci\'on de un requerimiento de cambio
		\item Aprobar o desaprobar el requerimiento
		\item Verificaci\'on, imlementaci\'on y liberaci\'on del cambio
	\end{enumerate}
	El plan identificar\'a los registros que se usan para documentar estos pasos. 
\end{frame}
\begin{frame}
	\frametitle{Auditorias y revisiones de la configuraci\'on}
	El plan identifica que revisiones y validaciones se haran en el 
	proyecto. 
	Para cada auditoria el plan define: 
	\begin{enumerate}
		\item Su objetivo.
		\item Los AC que seran auditado/revisado
		\item El periodo de las tareas de auditor\'ia o revisi\'on.
		\item Los procedimientos para conducir la auditor\'ia/revisi\'on
		\item Los participantes
		\item Documentaciones requeridas para las auditorias. 
		\item Los procedimientos para dar seguimiento a problemas. 
		\item El criterio de aprobaci\'on de la auditor\'ia. 
	\end{enumerate}
\end{frame}
\subsection{Programa de SCM}
\begin{frame}
	\frametitle{Programa de SCM}
	El plan describe:
	\begin{itemize}
		\item la secuencia y dependencias entre las actividades de SCM
		\item la relaci\'on entre las actividades de SCM y los momentos
			clave o eventos del proyecto. 
		\item los momentos clave y la linea base que establescera
		\item las fechas de arranque y terminac\'on de las actividades
			de auditor\'ia y revisi\'on. 
	\end{itemize}
	Todos los momentos ser\'an expresados en fechas absolutas. 
\end{frame}
\subsection{Recursos de SCM}
\begin{frame}
	\frametitle{Recursos de SCM}
	Documenta las herramientas, t\'ecmocas equipo, personal y entrenamiento
	necesario para las atividades de SCM
\end{frame}
\subsection{Mantenimiento del plan}
\begin{frame}
	\frametitle{Mantenimiento del plan}
	Identifica:
	\begin{itemize}
		\item El responsable del monitoreo del plan
		\item La frecuencia de las actualizaciones del plan
		\item C\'omo se evaluan y aprueban cambios al plan
		\item C\'omo se comunican los cambios al plan. 
	\end{itemize}
\end{frame}
\section{sumario}
\begin{frame}
	\frametitle{Sumario}
    \begin{itemize}
	    \item El standard IEEE 828 define las secciones necesarias 
		    para un plan de configuraci\'on de software.
	    \item Este standard cumple con criterios necesarios definidos en 
		    otros standares: ISO9001, CMMi, ITIL
    \end{itemize}
\end{frame}
\begin{frame}
	\frametitle{Actividades de aprendizaje}
	Escoge \alert{una} de las actividades siguientes:
	\begin{itemize}
		\item Si existe en tu organizaci\'on o proyecto, estudia el
			plan de SCM, e identifica si corresponde a las 
			secciones descritas en IEEE828. 
		\item Compara las actividades descritas en CMMi para SCM con
			las descritas en IEEE828. 
		\item Compara las actividades descritas en ITIL para SCM con 
			las descritas en IEEE828. 
	\end{itemize}
\end{frame}

