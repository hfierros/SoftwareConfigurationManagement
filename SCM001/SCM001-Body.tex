

\title{Administraci\'on de la configuraci\'on de Software}

\subtitle{Primera sesi\'on: Conceptos b\'asicos}

\pgfdeclareimage[height=0.5cm]{university-logo}{univa.png}
\logo{\pgfuseimage{university-logo}}

\titlegraphic{\includegraphics[height=1.5cm]{univa.png}}


\author{H\'ector Jos\'e Fierros L\'opez }


\date{Mayo 2017}

\subject{Talks}

\AtBeginSection[]
{
\begin{frame}<beamer>
\frametitle{?`Donde estamos?}
\tableofcontents[currentsection,currentsubsection]
\end{frame}
}


\begin{document}
\theoremstyle{definition}
\newtheorem{definicion}{Definici\'on}

\begin{frame}
\titlepage
\end{frame}

\begin{frame}
\frametitle{A cubrir hoy\ldots}
\tableofcontents
\end{frame}


\section{Introducci\'on}

\subsection[Sobre la clase]{Sobre la clase}

\begin{frame}
\frametitle{Administraci\'on de la configuraci\'on de software}
\framesubtitle{Objectivos.}

\begin{itemize}
  \item Conocer los conceptos de administraci\'on de configurac\'on de software.
  \item Conocer herramientas para la adminitraci\'on de configuraci\'on de
  software.
  \item Dise\~nar un proceso de administraci\'on de configuraci\'on de software 
  que responda a problem\'aticas espec\'ificas y sea susceptible de obtener
  determinada certificaci\'on. 
\end{itemize}
\end{frame}

\begin{frame}
\frametitle{Administraci\'on de la configuraci\'on de software}
\framesubtitle{Temas}

\begin{itemize}
  \item Administraci\'on de la configuraci\'on. 
  \item Proceso de administraci\'on de la configuraci\'on.
  \item Herramientas de la administraci\'on de la configuraci\'on
  \item Estandares y certificaci\'on de proceso de la administraci\'on.
\end{itemize}

\end{frame}
\begin{frame}[fragile]
\frametitle{Administraci\'on de la configuraci\'on de software}
\framesubtitle{Calificaciones}
\begin{tabular}{|p{8cm} |c|}
\hline
Concepto & Peso \\
\hline
Reportes de lectura e investigaci\'on & 20\% \\
Reportes de an\'alisis y ejercicios en clase & 20\% \\
Reportes de pr\'acticas &  10\% \\
Elaboraci\'on de un plan de Administrac\'on de Configuraci\'on de Software &
50\% \\
\hline
\end{tabular}
\end{frame}

\begin{frame}
\frametitle{Administraci\'on de la configuraci\'on de software}
\framesubtitle{Bibliograf\'ia}
\begin{itemize}
  \item Bruegge-Dutoit, Object Oriented Software Engineering (Chapter 13)
  \item IEEE Standard for Software Configuration Management Plans, IEEE Standards 
Board
  \item IEEE Guide to Software Configuration Management, IEEE Standards Board 
  \item P\'aginas de web de las herramientas: git, mercurial, cvs, rcs, etc. 
\end{itemize}

\end{frame}
\begin{frame}
\frametitle{Algunas reglas b\'asicas}
\begin{itemize}
  \item Celulares en silencio o vibrar
  \item Paso de lista al final de la clase. 
  \item No redes sociales. 
\end{itemize}
\end{frame}
\subsection{Acerca del maestro}
\begin{frame}
\frametitle{Sobre el maestro}
H\'ector Fierros, MC en Computer Science por UHCL

Desarrollador de SW desde 1991. 

Trabajando actualmente en Toshiba, desarrollando OS 4690
\end{frame}

\section{Conceptos b\'asicos de Administraci\'on de configuraci\'on de software}
\begin{frame}
\begin{exampleblock}{}
  {\large ``Cambia lo superficial\ldots cambia tambien lo profundo\ldots''}
  \vskip5mm
  \hspace*\fill{\small--- Todo cambia, Julio Numhauser}
\end{exampleblock}
\end{frame}
\subsection{?`Que?}
\begin{frame}
\frametitle{?`Que es la Administraci\'on de configuraci\'on de software?}

\begin{definicion}
La \alert{Administraci\'on de configuraci\'on de software} (SCM por sus siglas
en ingl\'es) es el proceso dedicado a la controlar los cambios que un producto
de trabajo puede tener. 
\end{definicion}

\end{frame}
\subsection{?`Por Que?}
\begin{frame}
\frametitle{?`Por que es necesario el SCM?}
\begin{center}
\includegraphics[scale=0.3]{MalDiscoDuro.png}
\end{center}
\end{frame}

\begin{frame}
\frametitle{?`Por que es necesario el SCM?}
\begin{center}
\includegraphics[scale=0.3]{CambiosAProduccion.png}
\end{center}
\end{frame}

\begin{frame}
\frametitle{?`Por que es necesario el SCM?}
\begin{center}
\includegraphics[scale=0.3]{BorraCambiosDeOtros.png}
\end{center}
\end{frame}

\begin{frame}
\frametitle{?`Por que es necesario el SCM?}
\begin{center}
\includegraphics[scale=0.3]{NewCompilerVersion.png}
\end{center}
\end{frame}

\begin{frame}
\frametitle{?`Por que es necesario el SCM?}
\begin{itemize}
  \item Manejo simultaneo de varias versiones de un producto de software
  tiempo.
  \begin{itemize}
    \item Producci\'on y desarrollo
    \item M\'ultiples versiones soportadas.
    \item M\'ultiples sistemas y ambientes soportados. 
    \item Versiones especiales. 
  \end{itemize}
  \item Multiples personas trabajando de manera simultanea en la misma base de
  c\'odigo. 
\end{itemize}
\end{frame} 

\begin{frame}
\frametitle{El prop\'osito del SCM}
Los productos de software son entidades evolutivas. Parten de una base, y existe
un desarrollo incremental. 
\begin{block}{SCM}
El Prop\'osito del SCM es coordinar los diferentes actores dentro del desarrollo
de un producto, a trav\'es de identificar art\'iculos de configuraci\'on,
administrar y controlar cambios, y poder auditar estos cambios.
\end{block}
\end{frame}

\subsection{?`Como?}
\begin{frame}
\frametitle{Actividades de SCM}
Seg\'un Bruegger las actividades de SCM se pueden definir como: 
\begin{itemize}
  \item Identificaci\'on de art\'iculos de configuraci\'on
  \item Administraci\'on de promociones
  \item Administraci\'on de lanzamientos
  \item Administraci\'on de ramas
  \item Administraci\'on de variantes
  \item Administraci\'on de cambios
\end{itemize}
\end{frame}

\begin{frame}
\frametitle{SCM seg\'un IEEE1042}
Seg\'un IEEE 1042 SCM se trata de: 
\begin{description}
\item[Identificaci\'on de art\'iculos de configuraci\'on] Los componentes del
sistema se identifican y etiquetan de manera \'unica. 
\item[Control de cambios] Cambios al sistema son controlados de manera que se
asegura la consistencia del sistema, y que esta se alinea con las metas del
sistema. 
\item[Contabilizaci\'on de estado] El estado de cada elemento individual del
sistema se almacena, de tal manera que es posible distinguir versiones de cada
elemento en particular. 
\item[Auditorias] Las versiones seleccionadas para ser liberadas son validadas
para asegurar que esten completas, consistentes, y cumplan con la calidad
requerida del producto.
\end{description}
\end{frame}
\begin{frame}
Algunos autores indican que la definici\'on de IEEE necesita tambi\'en:
\begin{description}
\item[Administraci\'on de construcci\'on o manufactura] Manejo de las
herramientas para poder construir el sistema final. 
\item[Administraci\'on de procesos] Administraci\'on de politicas y documentos
relacionados con el proceso, as\'i como la distribuci\'on de informaci\'on con
respecto al sistema. 
\end{description}
\end{frame}
\subsection{?`Quien?}
\begin{frame}
Los roles en el SCM son: 
\begin{description}
\item[Administrador de configuraci\'on] El responsable de identificar y manejar
los art\'iculos de configuraci\'on.
\item[Miembros del consejo de control de cambios] Responsables de aprobar o
rechazar requerimientos de cambio. 
\item[Desarrolladores] Responsables de realizar los cambios a los art\'iculos de
configuraci\'on
\item[Auditores] Encargado de seleccionar los cambios para la liberaci\'on del
producto. 
\end{description}
\end{frame}
\subsection{T\'erminolog\'ia}
\begin{frame}
\frametitle{Art\'iculos de configuraci\'on}

\begin{definicion}
Un \alert{Art\'iculo de configuraci\'on} es un producto de trabajo que es
tratado como una unidad. 
\end{definicion}

Ejemplos: 
\begin{itemize}
  \item El documento con requerimientos
  \item C\'odigo fuente
  \item Casos de uso
  \item Herramientas de compilaci\'on
  \item Diagramas de objetos y relaciones.
  \item Casos de pruebas 
  \item \ldots
\end{itemize}
\end{frame}

\begin{frame}
\frametitle{Art\'iculos de configuraci\'on agregados}
Un art\'iculo de configuraci\'on es como definici\'on una unidad. Sin embargo,
esta puede ser altamente compleja. 

Algunos autores hacen una diferencia entre art\'iculos de configuraci\'on
simples (por ejemplo, un archivo de c\'odigo fuente), y art\'iculos de
configuraci\'on complejos, (por ejemplo un sistema operativo).

A estos \'ultimos se les conoce como articulos de configuraci\'on conglomerado
(aggregate).
\end{frame}
\begin{frame}
\frametitle{Requerimiento de cambio}

\begin{definicion}
Un \alert{requerimiento de cambio} es un reporte formal que tiene el prop\'osito
de pedir un cambio en la configuraci\'on del sistema. 
\end{definicion}
El sistema de administraci\'on de cambios da la gu\'ia de como debe ser un
requerimiento de cambio. Por ejemplo, en un sistema militar americano, tiene que
sujetarse al MIL Std 480. En un sistema de desarrollo informal puede limitarse a
un mensaje instantaneo o a un correo. 
\end{frame}

\begin{frame}
\frametitle{Versiones}

\begin{definicion}
Una \alert{versi\'on} es el identificador de estado de un art\'iculo de
configuraci\'on. 
\end{definicion}
Al definir una versi\'on de un conglomerado, esta identifica las versiones
individuales de todos sus componentes. 
\end{frame}

\begin{frame}
\frametitle{Algunas versiones especiales}
\begin{description}
\item[Promoci\'on (promotion)] Una versi\'on que esta disponible internamente en
la organizaci\'on
\item[Lanzamiento (release)] Una versi\'on externa a la organizaci\'on
\item[Punto de referencia (baseline)] Una versi\'on que fu\'e acordada como un
la inicial por los participantes del proyecto.  
\end{description}
\end{frame}
\section*{Sumario}

\begin{frame}
\frametitle<presentation>{Sumario}

\begin{itemize}
  \item Definici\'on de Administrac\'on de Configuraci\'on de Software.
  \item Conceptos b\'asicos: 
  \begin{itemize}
    \item Art\'iculos de configuraci\'on
    \item Versiones
    \item Liberaci\'on
  \end{itemize}
\end{itemize}
\end{frame}
\begin{frame}
	\frametitle{Actividades de aprendizaje}
	\begin{itemize}
		\item Consulta la definici\'on de administraci\'on de la configuraci\'on de acuerdo a:
			\begin{enumerate}
				\item CMMi
				\item ITIL
				\item ISO 9001
				\item IEEE std 828/1042
			\end{enumerate}
		\item Analiza las diferencias entre \'estas definiciones.
	\end{itemize}
\end{frame}
