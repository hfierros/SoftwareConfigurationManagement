
\theoremstyle{definition}
\newtheorem{definicion}{Definici\'on}

\begin{frame}
\titlepage
\end{frame}

\begin{frame}
\frametitle{A cubrir hoy\ldots}
\tableofcontents
\end{frame}

\section{Herramientas de SCM}
\subsection{Tipos de las herramientas de software}
\begin{frame}
	\frametitle{Prop\'osito de las herramientas}
	El plan definir\'a las herramientas que se usar\'an.
	
	Es posible llevar el proceso "a mano" siempre y cuando se defina un 
	mecanismo de manejo de versiones. (Por ejemplo, incluyendo la versi\'on
	en el nombre del archivo, usando banderas de cambio en el c\'odigo, etc)

	Sin embargo existen herramientas de software que facilitan esta tarea
\end{frame}
\begin{frame}
	\frametitle{Las funciones de el SW SCM}
	Una herramienta de SCM debe:
	\begin{itemize}
		\item Manejar la identificaci\'on de los AC y sus versiones.
		\item Facilitar la colaboraci\'on entre los desarrolladores.
	\end{itemize}
	Una herramienta de SCM puede verse como una BD. 
\end{frame}
\begin{frame}
	\frametitle{Controladas localmente}
	Los art\'iculos de configuraci\'on y los deltas estan almacenados de 
	manera local en el sistema. 

	Los desarrolladores tienen que conectarse a este sistema, y 
	trabajar ah\'i.

	Ejemplos: 
	\begin{itemize}
		\item RCS
		\item SCCS
	\end{itemize}
\end{frame}
\begin{frame}
	\frametitle{Cliente/Servidor}
	Los art\'iculos de configuraci\'on son almacenados en un servidor 
	y los desarrolladores se conectan a trav\'es de un cliente. 
	Los cambios son coordinados por el software servidor. 
	
	Ejemplos: 
	\begin{itemize}
		\item CVS
		\item Subversion
		\item Accurev
		\item ClearCase
		\item Perforce
		\item VSS
	\end{itemize}
\end{frame}
\begin{frame}
	\frametitle{Modelo distribuido}
	Cada desarrollador trabaja en su propio repositorio local, y los 
	cambios y modificaciones son compartidos en pasos separados. 

	Ejemplos:
	\begin{itemize}
		\item Mercurial
		\item git
		\item arch
		\item Fossil
		\item VS Team Service
		\item TeamWare
	\end{itemize}
\end{frame}
\section{git}
\begin{frame}
	\frametitle{git}
	Git fu\'e desarrollado a f\'in de sustituir BitKeeper.
	Sus objetivos son:
	\begin{itemize}
		\item Velocidad
		\item Simplicidad
		\item Desarrollo no lineal
		\item Distribuido
		\item Manejo de proyectos grandes.
	\end{itemize}
\end{frame}
\begin{frame}
	\frametitle{Usando git en un SCM}
	Preparando un ejemplo muy sencillo: 
	\begin{itemize}
		\item Creamos un programa muy sencillo en C (hello world?)
		\item Crea un archivo de readme.txt 
		\item Crea un subdirectorio, donde pondras otro archivo de texto mas, llamado doc.txt. 
	\end{itemize}
	Estos tres archivos representan nuestros art\'iculos de configuraci\'on. 
\end{frame}
\begin{frame}
	\frametitle{Iniciando el baseline}
	Al instalar git, en windows se instalan tres comandos en el men\'u:
	\begin{description}
		\item [git-bash] Es un shell tipo bash de unix, con varios comandos de unix disponibles. 
		\item [git-cmd] Es un shell tipo CMD de windows, donde el path incluye los ejecutables de git. 
		\item [git-gui] Un interface gr\'afico para ver el trabajo con git. 
	\end{description}
	En linux y macOS, los comandos est\'an disponibles a trav\'es del shell. 
\end{frame}
\begin{frame}
	\frametitle{Configurac\'on de GIT}
	Despu\'es de instalar GIT, es recomendable: 
	\begin{itemize}
	\item Configurar tu nombre. \texttt{git config -global user.name "Juan Paco Pedro Pablo de la Mar"}
	\item Configurar tu correo. \texttt{git config -global user.mail "jppdlm@minombre.com"}
	\end{itemize}
	Estos datos se usar\'an cuando hagas promociones de c\'odigo. 
\end{frame}
\begin{frame}
	\frametitle{Creando el baseline de manera local}
	Para crear el baseline:
	\begin{itemize}
		\item En git-bash cambiate al directorio donde estan tus archivos. 
		\item Escribe el comando \texttt{git init}. Esto crea un repositorio local vacio de git.
		\item Usa el comando \texttt{git status} entre comandos. 
		\item Escribe el comando \texttt{git add .} Esto a\~nade todos nuestros archivos al repositorio de git.
		\item Escribe el comando \texttt{git commit}. Esto promueve los archivos del baseline en el repositorio. 
		\item El \texttt{git commit} lanza una ventana de vim, donde escribiras la raz\'on del commit.  
	\end{itemize}
	Y nuestro baseline esta creado. 
\end{frame}
\begin{frame}
	\frametitle{Entendiendo el repositorio}
	\begin{itemize}
		\item El comando \texttt{git ls-tree -r HEAD} da un listado de los archivos.
		\item El comando \texttt{git log} da una historia del sistema. 
		\item La meta-informaci\'on sobre el repositorio se almacena en el directorio \texttt{.git}
	\end{itemize}
\end{frame}
\begin{frame}
	\frametitle{Haciendo cambios: Nuevos archivos}
	\begin{itemize}
		\item Crea un nuevo archivo (test.txt) en el arbol. 
		\item Usa el comando \texttt{git status}. Muestra los cambios en el arbol. 
		\item Usa el comando \texttt{git add test.txt}. Pasa la parte al control de git. 
		\item Usa el comando \texttt{git status}. Muestra las partes listas para promover. 
		\item Usa el comando \texttt{git commit}. Hace la promoci\'on. 
	\end{itemize}
\end{frame}
\begin{frame}
	\frametitle{Haciendo cambios: Modificando archivos}
	\begin{itemize}
		\item Modifica uno de los archivos en el arbol. 
		\item Usa el comando \texttt{git status}: Muestra los cambios en el arbol.
		\item Usa el comando \texttt{git add -A}: Pasa los cambios al control de git. 
		\item Usa el comando \texttt{git status}: Muestra las partes listas para promover.
		\item Usa el comando \texttt{git commit}: Hace la promoci\'on.
	\end{itemize}
\end{frame}
\begin{frame}
	\frametitle{Checando la historia}
	\begin{itemize}
		\item El comando \texttt{git log} da un hist\'orico de los cambios
		\item El comando \texttt{git show} da un vistazo de los cambios en cada commit
		\item El comando \texttt{git diff} saca las diferencias en los archivos antes de hacer commit. 
	\end{itemize}
\end{frame}

\begin{frame}
	\frametitle{Branches}
	\begin{itemize}
		\item Cada commit cambia el estado del repositorio.
		\item Un branch es un apuntador a uno de los diferentes estados. 
		\item Hay una rama principal llamada \emph{master}
		\item El comando \texttt{git branch} sin parametros, lista las branches.
		\item El comando \texttt{git branch nuevarama}, crea un nuevo branch, llamado nuevarama.
		\item El comando \texttt{git checkout nuevarama} hace que el nuevo branch sea el que recibe los commits. 
		\item El comando \texttt{git merge} Une diferentes branches. 
	\end{itemize}
\end{frame}
\begin{frame}
	\frametitle{Remotos}
	La parte mas poderosa del git es la capacidad de manejar datos distribuidos. \
	\begin{itemize}
		\item Un repositorio "local" en un servidor de archivos. 
		\item Un repositorio remoto, a trav\'es de:
			\begin{itemize}
				\item ssh
				\item ftp
				\item http
				\item git
			\end{itemize}
		\item El comando \texttt{git clone} clona un repositorio remoto a un directorio.
		\item Los comandos \texttt{git pull} y \texttt{git push} "bajan" y "suben" los cambios al remoto.
		\item El comando \texttt{git remote} da informaci\'on sobre los remotos. 
	\end{itemize}
\end{frame}
\begin{frame}
	\frametitle{GitHub}
	\begin{itemize}
	\item GitHub es una plataforma de desarrollo colaborativo basada en git. 
	\item Permite el manejo de proyectos p\'ublicos con algo de funcionalidad de redes sociales. 
	\item Permite la creaci\'on de un Wiki por cada proyecto. 
	\item Para usuarios de paga, permite creaci\'on de proyectos privados. 
	\end{itemize}
\end{frame}

\begin{frame}
	\frametitle{Actividad de aprendizaje}
	\begin{itemize}
		\item{Usando git identifica\ldots}
			\begin{itemize}
				\item \ldots como instalar 
				\item \ldots que comando usar para crear un baseline
				\item \ldots que comando(s) usar para hacer una promoci\'on
				\item \ldots que comandos usar para hacer un branch
			\end{itemize}
		\item{repite usando mercurial, subversion y/o otro sistema de control de versi\'on}
	\end{itemize}
\end{frame}

